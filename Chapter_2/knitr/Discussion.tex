\section{Discussion}

My model calculates mortality-minimizing decisions for migrants moving through a landscape of a small and large stopover site on post-breeding migration. I use the model to develop an understanding of how food and safety trade-off in stopover site selection. Few models have examined behaviour on southward migration, but my model uses methodology comparable to the many developed for northward migration. I generated patterns of lengths of stay, mass, and response to predator population that match many of the observed migratory patterns in western sandpipers. The model also illuminates potential causes for shifts between sites and reductions in numbers using a region.

The main model tradeoff is between food and safety at the local sites. Carrying heavier fuel loads increases migrant vulnerability \citep{burns_effects_2002}, and this effect is strong at the small site, where danger is higher. Migrants at the small site depart lighter, despite being at a more food rich location. As I adjusted the ratio of food to safety I found that the distribution between the two sites has sudden transitions in both food and safety after which the simulated birds are found primarily at a single site. Such transitions suggest that sharp changes in observed counts at stopover sites should be expected if conditions change beyond a region of stability, and the distribution across sites with varying food and safety should be more aggregated at a few ``ideal'' sites rather than being more normally distributed.

Timing within the predator landscape also drives much of the migrant behaviour in the model. The arrival of migratory peregrine falcons has been described as a broad-front phenomenon, where the arrival of the falcons is a given region occurs quickly and then presence remains high as the falcon wave passes through \citep{Ydenberg2007b,Worcester2008,lank_effects_2003,Hope2011,Hope2014}. A migrant's proximity to the arrival of this falcon front, when daily probability of predator presence, $\theta(t)$, is increasing at its fastest rate, drives many of the seasonal patterns observed in the model. Migrants ahead of the inflection point in $\theta(t)$ ($-B_0/B_1$; August 12th) make much greater use of the small site and depart lighter, while birds under the falcon front avoid the small, dangerous site and depart with higher fuel loads, allowing them to jump ahead of the falcon front. As the timing of the falcon front has been shown to vary from year to year, it would be expected that migrant behaviour should differ in years with earlier or later falcon arrival \citep{Niehaus2006}.

% The decisions to stay, move or depart are driven by the marginal change in survival and fuel load over the between days. When the change in the number of falcon attacks between two days increased with the seasonal change in predator presence ($\theta(t)$), the relative benefit of staying another day to add more fuel decreases. Migrating sandpipers must make these types of tradeoffs in real time, but the model allows these tradeoffs to be strictly defined and identified.  

% Adult and juvenile migrants, with no programmed differences other than timing of arrival, have differences in stopover site usage within the model.  Under the same baseline conditions adults are more likely to use and spend longer at small site while a month later juveniles stay for less time and are found in lower numbers. 

Sensitivity to predation and food has been considered in other migratory models. \citet{Taylor2007} found a similar shift away from small sites with increased predation during northward migration, though their transition was more gradual, and the relative safety of the site types did not change. The migration of barnacle geese northward showed a similarly dramatic shift in migratory onset with an increase in predation \citep{Jonker2010}. The response curve was similar to that from a more general migratory model where a small change in the ``foraging intensity-dependent predation'' at a stopover site led migrants at the previous site to rapidly increase the amount of fuel they loaded \citep{Weber1998}. It is symptomatic of DSVMs to have sudden transitions in decisions as state variables change \citep{ClarkMangel2000}, but what drives such transitions can tell us about what drives the system.

% In my model the decision to move sites and to depart is primarily driven by both fuel load and date.  The departure decision is also driven by fuel load and date but the trend in optimal departure fuel load across the season was smaller relative to the local decision (\autoref{fig:decision}). While departing with higher fuel loads should be beneficial, the cost to safety of carrying higher fuel loads and the increasing danger of remaining to fuel longer limits the realization of higher fuel loads. The decision to move between sites is affected by fuel load for any given date, but the temporal aspect of predation danger, $\theta(t)$ affects the optimal movement fuel load, with fuel load at which migrants should switch sites decreasing across the season. 

\subsection*{Comparison of model outputs to field data}

The model was able to broadly reproduce observed differences in body masses of western sandpipers on southward migration between a small and large stopover site. Both the model output and the capture data show birds at the small site (Sidney Island) were lighter than on the large site (Boundary Bay). The model results suggest that the mechanism for this pattern is that for light birds the benefit of increased fuel loading outweighs the increased risk of capture by predators. As birds load fuel, the vulnerability to capture, $\psi(x,s)$, increases such that the marginal benefit to survival of quicker fuel loading is eventually smaller than the reduced escape performance at higher fuel loads. As the daily number of attacks increases through the season, the fuel load at which the benefit from the small site's the higher fuel loading rate decreases. %The capture data does not have the coverage to explore this, but there is some suggestion that later Sidney Island adults and juveniles may be lighter than earlier birds (\autoref{fig:baseline-res}b).

The intraseasonal trend in lengths of stays for the small site produced by the model approximates that observed in mark-recapture data from a small site \citep{Hope2011}. This similarity is reduced for migrants, especially adults, given prior knowledge in their initial settlement decisions. A priori, we might have expected the opposite outcome, that is, that prior knowledge would provide a better fit for adults than juveniles. The optimality of both sites appears to be more equal for light birds earlier in the migration period, so the cost of moving between sites is greater than the benefit from switching locations. As the falcon front arrives, the cost in vulnerability of being heavy at the dangerous site means migrants above a given fuel load must be at the safe site, no matter their current location. Adults appear to be able to optimize their migration no matter which site they arrive at, while for juveniles, arrival at the better of the two site types may allow for higher chances of surviving migration.

 % When migrants do not have prior knowledge the pattern is very similar between model and mark-recapture data, however with prior knowledge the similarity is reduced. Juvenile trends still roughly match the observed trends with lengths of stay dropping over the season, but the observed trend in the mark-recapture analysis showed an increase across the adult migratory period \citep{Hope2011}, while when prior knowledge is included our model fails to replicate this at the small site. 

% My model shows the opposite, with decreases in departure fuel loads within the adult migratory season \autoref{fig:decision}. The current model structure does not have a variable to account for future speed of migration, though it could be a potential future modification of the terminal fitness function. The discrepancy could also be the result of differing techniques in estimating daily residence probability between the model output and the mark-recapture model. My model outputs are the true estimate of individual birds being present in the next day, whereas \citet{Hope2011} presented the estimated trend in daily residence probability estimated weekly while also estimating resighting probabilities.

I was able to replicate the observed changes in migratory tactics with increasing predator numbers that was observed in captured western sandpipers \citep{ydenberg_western_2004}. Declining sandpiper masses as the number of predators increases should be broadly expected, because the marginal cost to scape performance from adding fuel is higher when the number of attacks increases. To replicate the observed pattern in capture data of a larger decrease in masses at Sidney Island than Boundary Bay as predation increased required the small site's impact on capture probability (predation danger, $\tau(s)$), to shift as falcon abundance changed. In the model I included $\tau(s)^F$, but if only $\tau(s)$ is used then the decline in masses is equivalent between sites as predation increases. It seems likely that the impact of increases in predation is disproportionally felt in more dangerous habitat, as has been observed in other systems \citep{Brown2004}. 

Increasing the predator population produced changes in overall lengths of stays at the small site that matched those observed at Sidney Island \citep{ydenberg_western_2004}. There are no comparable mark-recapture estimates of lengths of stay at Boundary Bay or any other large site on southward migration, but the model's prediction of an increase in lengths of stay as the predator population increases is consistent with migrants trading off the future benefit of departing with higher fuel loads against the current exposure to predation.  By spending less time in the most dangerous habitat migrants are still able to remain locally and load fuel and retain a higher daily survival probability. The model demonstrates that migrants reduce exposure to dangerous habitat as predator populations increase.

The shifts in tactics produced by the model are slightly different than those proposed by \citet{ydenberg_western_2004}. They found that as predator populations recovered the number of birds using Sidney Island did not change, only their lengths of stay. In my model, migrants reduced their lengths of stay and were generally lighter when predator populations increased, but the true numbers of birds using the site also declined (\autoref{fig:count-vs-true}), especially when prior knowledge was added to arrival. It is not an unreasonable expectation that as a habitat becomes more dangerous an increasing number of birds would avoid that habitat. In the model, heavier and later migrants switch to the other site or avoid it if they arrive with prior knowledge, rather than depart onwards sooner. The final years of the \citet{ydenberg_western_2004} do show some decline in the true numbers using the Sidney Island site, which could indicate some shift of birds away from using the site at all. The model results suggest migrants may be likely to arrive at Sidney Island randomly and then make the decision as to move or depart.

 A potential cause of the difference between model output and western sandpiper decisions is the amount of movement between Sidney Island and Boundary Bay. There is substantial movement between the large and small site in the model. Without prior knowledge movement occurs for both adults and juveniles, though with prior knowledge, movement is only from the small to large site in juveniles (\autoref{fig:path-birds}). While \citet{Ydenberg2002} show there is some movement between the sites, it is potentially less than is observed in the model. This could mean that the model is overestimating the propensity of migrants to shift between the two sites, perhaps by omitting a transition cost above and beyond the energetic cost.  Early models of optimal migration strategies produced substantially different results when settlement costs at a new site were added \citep{Alerstam1990}. 

% Western sandpipers may use only either Boundary Bay or Sidney Island, but what is classified as a movement from small to large site in the model could be a movement from Sidney Island to another large site further south in the Salish Sea.  Sidney Island lies 50-60 km to the southwest of the Fraser River Delta. A flight from Sidney Island to Boundary Bay would be away from the main direction of migration and there are comparably large sites an equal distance from Sidney Island that lie further to the south. Alternatively, a movement from Boundary Bay to a small site, could just as easily be to another small site in the region of which Sidney Island is just one \citep{Pomeroy2008a}.


%A flight from Sidney Island to Boundary Bay would be away from the main direction of migration and there are comparably large sites an equal distance from Sidney Island that lie further to the south. So it seems possible to me that movements from one site type to another would not always show up in a dataset that did not include other comparable large or small sites. Additionally if some proportion of the population arrived without prior knowledge, it would likely lead to lower shifts in total numbers of birds using the sites as conditions change.

% While the model may be limited in a full comparison with Boundary Bay and Sidney Island, it does not limit it's usefulness. 

The model predicts that, in general, as the region becomes more dangerous when predator numbers increase, the overall pattern is that small sites should have a smaller proportion of birds that stay for shorter periods and are lighter, while larger sites should increasingly be utilized and have little change in masses, but longer lengths of stays. The model also predicts that lighter juveniles departing a small site should remain locally at a safer location before departing onward, whereas adults appear to depart onward from whichever site they arrive at. A test of this prediction would be to collect movement patterns of migrating western sandpipers between sites on southward migration. With the development and expansion of cheaper tracking technology such as the Motus Wildlife Tracking System \citep{MOTUS_2017} a field project tracking the movement of western sandpipers within the Salish Sea would allow us to explore how movements contrast with model results. Capturing birds at Sidney Island and tracking their movements on departure would allow us to see if they are moving to a nearby large site, as the model predicts, or if they are departing onward to more distant sites such as Gray's Harbor. 

% Modifying the seasonal trend in predation danger reduced the importance of peregrine falcon migration on migrant behaviour. The baseline model used the daily probability of falcon observation in the Fraser River Estuary as a proxy for the seasonal trend in relative probability of predator presence at a site. Other studies have used the cumulative passage abundance as an index of falcon migration \citep{lank_effects_2003,Niehaus2006,Worcester2008}. Their measure more closely matches my modified $\mu_t$ as their predator index continues to increase through August and September while in my original index the rate of increase in $\mu_t$ is highest around the start of August and slows through the month. It is likely this modified shape that is most important and not just the change in absolute values of $\mu_t$ that prompt juveniles to change behaviour in the modified model. While it has been assumed that migrant sandpiper vulnerability dramatically increases as migratory peregrines begin to arrive, it could be that for an individual sandpiper this increase is more gradual as the total number of peregrines in the region increase.

\subsection*{Assumptions of migrant knowledge}

Models must always make assumptions as to how much knowledge of the system individuals have. Perfect knowledge allows us to model the behaviour of under ideal conditions and make an inference as to the true level of knowledge. Mechanistic models can illuminate how patterns of behaviour emerge with much narrower knowledge \citep{Bracis2018}. My model provides migrants with knowledge of the food and safety at both sites. There is some stochasticity in the estimates, but migrants make decisions on the accurate expectation of the values. They also have accurate knowledge of the costs and benefits of loading fuel and of the temporal trend in predation danger. I expect selection to favour migrants that behave as if they have the knowledge of the relationship between survival probability and differences in fuel load or stopover length.

The location of a migrant's initial arrival is a decision the baseline model assigns randomly. Random arrival is very likely unrealistic as even a naive migrant should be able to distinguish between site types as they arrive and make a non-random decision as to where to land. Random arrival could be realistic where migrants are forced to land immediately, e.g. due to their condition or weather, but this is not generally the case in the southward western sandpiper migration, as most migrants appear to arrive with some fuel reserves \citep{Butler1987}. Allowing migrants to settle at an optimal location allowed us to model migrant arrival more realistically, but even random arrival allowed us to examine responses to changing conditions while excluding the initial, but potentially separate decision of first arrival location.


% \subsection*{Foraging intensity: to model or not to model?}

% Adding foraging intensity to the model allowed migrants to accept greater predation danger through behavioural adjustment. Numerous behavioural studies have shown individuals will adjust foraging strategies based on their current location's danger \citep[e.g.]{brown1999vigilance,Brown2004,Kotler2007,Kotler2017,pomeroy_tradeoffs_2006,Brown1992,Bedoya-Perez2013}. In my modified model, daily counts were very similar with and without foraging intensity. Juvenile counts were slightly higher at the large site when foraging intensity was included, but otherwise they were very comparable. The main difference between the models was that with foraging intensity, migrants were about 1 gram heavier at both sites for adults and at the large site for juveniles. Migrants are clearly able to reduce their vulnerability to a successful predator attack by reducing their foraging intensity. Later in the season, as the number of predator attacks increases, the required reduction in foraging intensity is too great to incentivize being heavier at the small site. 

% The result of reduced foraging intensity and staying until they are heavier is that lengths of stay were longer in the model with foraging intensity. This difference was greatest for juveniles at the large site but present for adults at both sites as well. \citet{brown1999vigilance} explored how vigilance should vary with the value of energy, effectiveness of vigilance, predation danger, and food abundance. He found that the changes in the effectiveness of vigilance strongly shaped the optimal vigilance, but mostly when effectiveness of vigilance was low and food availability should affect this relationship. How effective vigilance is in the avian migratory system remains open to debate. Western sandpipers avoided areas where the effectiveness of vigilance would be reduced (close to cover) but increased vigilance when they did use these areas \citep{pomeroy_tradeoffs_2006,pomeroy_experimental_2006,Hope2014}. Adjusting the benefit of vigilance should shift how migrants respond to the fundamental food/safety tradeoff that the model explores.


% A key difference with these studies is that migrating individuals in my model are choosing the more dangerous habitat, suggesting under baseline conditions any adjustments in behaviour can compensate for increases in vulnerability due to habitat danger. The seasonal trend in predation danger was also compensated by reducing foraging intensity, allowing even later arriving juveniles to choose the small site. 

% The discrepancy in site usage between the original model and the one with foraging intensity could be driven by the food and safety parameters ($\mu_f$; $e_g$) or the form of \autoref{eq:mu_forage}. \citet{Jonker2010} used a slightly different mortality function (see their equation 4), which allows very much less compensation from condition dependent mortality risk (due to fuel load in their case) than our model for which even the heaviest birds ($x=1.0$) can reduce their overall predation vulnerability by 25\% by adjusting foraging intensity, whereas in the goose model a similar adjustment resulted in only a 1\% change in a heavy individual's overall mortality risk.  \citet{brown1999vigilance} explored how vigilance should vary with the value of energy, effectiveness of vigilance, predation danger, and food abundance. He found that the changes in the effectiveness of vigilance strongly shaped the optimal vigilance, but mostly when effectiveness of vigilance was low and food availability should affect this relationship. How effective vigilance is in the avian migratory system remains open to debate. Western sandpipers avoided areas where the effectiveness of vigilance would be reduced (close to cover) but increased vigilance when they did use these areas \citep{pomeroy_tradeoffs_2006,pomeroy_experimental_2006,Hope2014}. It could be that my model overestimates the benefit of vigilance, but reducing the benefit too much will eliminate the fundamental food/safety tradeoff that the model explores. 


% Overall, the basic behaviour of the model remains similar in the models with and without foraging intensity. Adding foraging intensity to the model meant our estimates of relative fuel loading would need adjusting to be accurate as it is possible the safe and large site could allow for higher daily loading rates, but are never realized because of the cost in safety. I decided to use the simpler model without foraging intensity as I could examine stopover decisions without modelling foraging intensity explicitly.

\subsection*{The effect of migratory distance}

\citet{Hope2011} explain the season increase in adult lengths of stay as strategy to increase their speed of migration and to stay ahead of migrating peregrine falcons. My model shows departure and local decisions are dependent on the amount of migration remaining (\autoref{fig:mig-dist-range-speed}). Under the baseline conditions, the expected speed of migration on departure does indeed increase through the adult migratory period and then decrease through the juveniles' as they prioritize current safety. However, the impact of different migratory distances is much stronger than any intraseasonal trend for a particular distance. Shorter-distance migrants appear to be prioritizing flight range, while longer distance migrants appear to select their departure fuel load to maximize future speed of migration. This result can be viewed under the risk allocation hypothesis \citep{Lima1999a} as longer-distance migrants will face sustained exposure to higher rates of predator attacks, while the model assesses shorter distance migrants being able to reach a safer non-breeding location in a single flight.

The model was not initially developed to explore the impact of future migratory distance on local decisions, but it does make some predictions that would be interesting to examine. A model of the entire migration including the decision to select the optimal non-breeding locations would be required to fully parse out the trade off between local decisions and future exposure to migration. While my model shows migrants with longer distance to migrate should take more risks locally to shorten the time they spend on migration, it is just as possible that migrants facing higher local predation danger or vulnerability should adjust their migratory distance to ensure survival through the non-breeding period. 


\subsection*{Juvenile migration: naive or optimal?}

My model generated distinct patterns of behaviour and usage for adults and juveniles with no programmed differences in their decisions other than arrival timing. Their temporal position alone shaped their decisions to depart lighter and remain for shorter durations at the small site. While age related differences are often attributed to naivety, development, lower social dominance, or doomed surplus \citep{newton2010migration}, examining the expected differences in behaviour based on their different conditions or life-history as a first step seems reasonable before defaulting to intrinsic age-specific differences in behaviour as an explanation. 


% The baseline conditions of the model did not predict juvenile usage at the small site very accurately. Adding prior knowledge to the forward simulation meant they would avoid the site, even as food and safety conditions were adjusted. The key factor driving this behaviour is the temporal trend in predation danger $\mu_t$. Beyond around model day 50 (August 9th) the daily change in predation danger was steepest, which likely drove juvenile migrant away from the small site. Adjusting the slope of $\mu_t$ ($B_1$) resolved this issue without adding any age-specific behavioural parameters and indeed with this adjustment juvenile mass patterns were more compatible to the field data when they arrived optimally compared with random arrival.

% Overall, the model shows that juvenile behaviour in the western sandpiper system can in often be explained by differences in their timing of arrival alone. Age-related differences based on developmental state, experience, or dominance are important elsewhere in most animal systems and may be here also, but my results suggest they may not be necessary to explain every age-related difference in behaviours.

%%%% YOU ARE HERE -----------------------------------------------------


\subsection*{Future potential model directions}

My model of a highly complex system was designed to be simple to understand and interpret. Factors missing from the model that have been shown to impact behaviour elsewhere are foraging intensity, seasonal changes in food abundance (See Chapter 3), and density dependent effects on both foraging and vulnerability to predation. There is no evidence that habitat availability on a large scale on southward migration is in any way constraining to shorebird populations. High enough densities would impact fuelling rates, but daily numbers are much lower on southward migration than on northward migration, while food abundance is higher \citep[Chapter 3]{Pomeroy2006a,drever_monitoring_2014}. Interference competition could be a source of reductions in fuelling rates with higher densities of migrants. Semipalmated sandpipers show evidence of scramble competition and scrounging in large flocks \citep{Beauchamp2014,Beauchamp2012}, suggesting that flocking may drive competitive interactions even if overall food availability is high. An interesting extension to the model would be to reduce the time step to hours or minutes and include density mediated interactions within flocks and how that should change across a tidal cycle. A comparison of these effects with those of dilution would be an interesting but separate topic. The time step of one day should average out many of these effects across a full tidal cycle. 

Dilution of vulnerability to predation is another aspect my model does not consider. At most stopover sites, flock sizes will be well above the level where changes in numbers will affect dilution. Most flock size effects on vigilance level out around 20-50 individuals \citep{Elgar1989,Lima1995,Roberts1996}. Very small sites could see an effect as they can host numbers below 100 individuals regularly, and modelling density effects should produce impacts at the smallest sites. However, for my model, I assume any dilution/competition effects of density are unimportant in this system and would require the model to increase in specificity, thereby reduce the generality of any results it produces. 

The impact of prior knowledge on my results highlighted the importance of the arrival decision in shaping migratory counts. Western sandpiper arrival site choice is likely shaped by model factors including migrant condition, and date, but the spatial path a sandpiper uses on arrival and location of conspecifics are additional factors that could shape where sandpipers first end up. Arrival decisions have been hard to study in the past, but with the increasing availability of inexpensive tracking technology future research should be able to illuminate how migrants choose their first arrival location after long flights. 

Adding a spatial dimension to the model would also allow for increased specificity in the predictions. As I mention above, my model was designed to be general and simple and adding spatial specificity requires increased parameterization, thereby reducing its generality. That said, a more spatially explicit model could examine an array of stopover sites of varying sizes as well as examine how factors such as arrival path affect stopover selection. For example, how good does a site further away have to be for a migrant to skip a mediocre site that they encounter first? These questions are explicitly of interest for migrants that cross a large barrier and are faced with a broad selection of potential stopover sites. Modelling migration upon departure for western sandpipers from the breeding ground could produce in some interesting insights to how migrants move from Alaska to their non-breeding grounds. Preliminary geolocator work suggests some western sandpipers migrate out to the Alaska Peninsula and then fly directly to California \citep{north2016state}. Understanding the state- and condition-dependence of this decision versus flying only to the Salish Sea would substantially increase our understanding of migratory behaviour at a larger spatial scale.

It is likely that that adding temporal, spatial resolution and density-related effects would make the model difficult to implement as a dynamic state variable model. An individual based model would be simpler to implement, but often faces the drawback of being specific to a single system and hard to parametrize. Nonetheless, as more detailed tracking information becomes available, the next step would be to use a modelling approach to understand in more detail some of the patterns observed in this model. This model uses mortality-minimization as a theoretical background and any model advancement should also start from a theoretical underpinning \citep{GrimmVolkerandRailsback2005}.

\subsection*{Uses of current model}


My simple model of passage through a landscape of stopover sites defines the tradeoffs migrants must make on southward migration and the effects these tradeoffs have on measured behaviour of migrants. Theoretical models have tended to avoid examining southward migration and my model adds some understanding to the specific system of western sandpipers and more broadly to shorebird migrations. The model can replicate observed differences in masses between sites and the response to predator abundance in masses and lengths of stays.  Box~\ref{box:sesa} shows the model can be modified to examine other systems, in this case broadly replicating the staging behaviour of semipalmated sandpipers by adjusting the timing of falcon arrival.

The next step is to run model scenarios and examine the observed patterns in counts at the two site types. \citet{Taylor2007} referred to behavioural indicators that could be indicative of population change or be used to rule out other explanations for census declines. Chapter 3 uses specific predictions from this simple model to assess the support for given scenarios from shorebird surveys of western sandpipers on southward migration.




















% I investigate how site conditions affected the decisions made on southward migration for migrants optimizing their chance of surviving migration. I developed a model of southward migrating shorebirds passing through a landscape of small and large stopover sites. The model was parameterized around Western Sandpiper migration through the Salish Sea in British Columbia. The model initially allowed migrants to load fuel, move between sites and depart on a daily basis. Sites varied in their potential fueling rate and predation danger. Decisions were affected by a the seasonal pattern in predator abundance \cite{Lank2003} and by differences in the conditions at the two site types.

% Adult and juvenile migrant passages occurred at a different points along the seasonal predator landscape, leading to differing usage of the small and large sites. Adults primarily used the small, food rich, but dangerous site, while juveniles made use of the large, safe, but food poor site. With freedom of their first move juveniles completely avoided the small site and often skipped the entire region. This difference highlights the importance of the seasonal trend in falcon abundance in shaping the movement of birds through a migratory landscape. These differences between adults and juveniles occur even with no other inherent differences between adults and juveniles programmed into the model, suggesting observed differences between adults and juveniles on migration can arise without the assumption of naivety. In this case, however it appears that patterns in model outputs are more realistic for juveniles without prior knowledge, as with it later migrants are very reluctant to use the small site.

% When the relative food and danger was adjusted at the large site the distribution of birds shifted quickly in two ways. Firstly, the primary site shifted between the small and large site as the danger or food at the large site was adjusted. This could be considered a true distribution shift as the optimal stopover site is changing. However, the second pattern that was observed was a shift in the number of birds using the optimal site. In this case, the preferred site remained the same, but birds shortened their length of stay or, if the model allowed, skipped the region entirely. While this skipping behaviour might be expected in the broader context of the migration, it was not expected to be observed on this local scale model. In the baseline model, increasing the food abundance at the large site initially had no effect on the distribution or skipping, but at a certain level ($e_{s=1}=0.5$) the birds shifted from being primarily at the small site to the large site. With further increases in food, skipping or early departures decreased and the abundance at the large site continued to increase, though only up to a point that allowed migrants to quickly load fuel and depart. 

% For adjustments in danger, there was a difference in response between adults an juveniles. For early migrants, changes in predation danger at the large site ($\mu_{s=1}$) led initially to an increase in skipping and a decrease in counts at the large site, followed by a distribution shift towards the small site. For later migrating juveniles, there was no distribution shift as the temporal pattern in predator abundance had made the small site undesirable. Instead increasing danger at the large site lead to skipping or early departures and a decrease in abundance at the large site. While adding foraging intensity or prior knowledge to the model changed some of the details of the distributions and the points at which they shifted, the general pattern remains in differing responses based on timing and distribution shifts occurring with changes in food while skipping or early departures occurring in response to changing danger. 

% Migration results -- 
% 	- Juveniles avoid dangerous sites, adults take advantage of increased food
% 	- Lengths of stay in the region were very short
% 	- Adding foraging intensity increased amount of danger migrants would accept
% 	- Allowing prior knowledge led to avoidance of one site
% 	- Adjusting food and safety showed distributions were stable up to a point then shifted rapidly.
% 	- Transition points occurred for both adjustments of food and danger 
% 	- When model allows, migrants will skip a region
	
% Results suggestive that both food and safety are important and that rapid transitions in distributions can occur due to small shifts in variables.

% Foraging intensity allows 

% Minor adjustments to the model allowed me to reproduce observed patterns in masses and lengths of stay at the sites used to parametrize the model. The observed pattern of heavier birds being captured at a larger stopover site was reproduced in the model by increasing the discount in safety for departure. The predator landscape shifts seasonally from north to south as migrating peregrine falcons move southward after their breeding season in the arctic \citep{worster}. Therefore it is not an unreasonable modification to make that migrant sandpipers on southward migration expect their future to be safer than current conditions. \textbf{This may not be true for juveniles, but as we discuss below, modeling juveniles and adults together does present some difficulties.}


% Juveniles were much less sensitive to adjustments in model parameters. They sit directly under the migration of their predator and as such any benefit in safety afforded them from using a larger site is taken advantage of. Adjustments to food, have almost no impact on their usage of the small site and increasing the danger at the large site, does not encourage many birds to switch to the small site. Across the adult migratory period the temporal aspect of predation danger, $\mu_t$, increases from less than 0.1 to about 0.25, while across the juvenile period it increases up to about 0.75, with the steepest slope occuring around the peak of juvenile arrival. It seems that their position on the predator landscape drives their decisions to be less driven by their state than by adults. Changes in site danger shift juveniles to depart lighter, rather than to adjust their stopover site selection.


% Comparisons to other models --
% 	WESA -- Taylor, Butler - Danger important - Distinct patterns
% 				-> Differences due to southward migration?

% The two other models of western sandpiper migration examine northward migration across a flyway, but do present some interesting contrasts. Clark and Butler found that realistic migratory patterns do not occur unless their is a latitudinal gradient in predation, with lower danger in the southern wintering sites. They suggest such a gradient could shape migratory strategies in sandpipers. 

% Taylor et al. show both food and danger impact migratory behaviour on migration, but do not explore how the relative measures between site types should affect patterns of migratory usage.


% 	Geese - Bauer and derivatives
% 		-> differences between models and what commonality is there
% 	Weber -> Used as the base for many migratory models - what differences in your model.
% 	Southward geese - Do they look at site selection?
% 	Fred and Dill  - They predict stability at certain ratio of Danger/Food - where is it and do you see a similar result?

% Importance -- highlights both food and safety, and that adjustments in foraging intensity can partially or completely compensate for increased danger. Model can reproduce simple patterns

% Where it fails -- Juvenile usage of small sites. Length of stay trends. Migrants cannot starve.
% 	- Lack of density dependence - is that an issue here? Dilution effect? Anything missing from this?
% 	- Little empirical support for fitness function. What alternatives could be run.
% 	- DSVM vs other model types
% 		* IBM
% 		* Flow models
% 		* Species distribution models
% 		* Energetic models
% 		* Optimal migration models

% Applications -- Run scenarios
% 	-- Compare to data
% 	-- understanding migratory counts
% 	-- Generality of results - SESA
% 	-- Can we understand migratory counts without knowing site/food accurately
% 	-- What about the missing (skipping) migrants? Important factor in model.
% 	-- Counts can change dramatically without a population change