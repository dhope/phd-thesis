\section{Introduction}
%\linenumbers
% \subsection*{Northward migration well modelled, southward less so.}
Long-distance avian migration has been well studied empirically and theoretically. While advances in tracking technologies have made empirical measures of migration much more precise, modelling movements and decisions on migration have provided some of our greatest insights into how the patterns we observe are shaped. Modelling efforts have focused on the migration towards the breeding ground, with the post-breeding migration receiving less study \citep{gallinat2015autumn}. 

Post-breeding migration features a mixture of age classes, with hatch year birds travelling on their first migration; variation in timing, direction, destination, and urgency amongst age, sex, and size classes; and an overall structure of external factors that differentiates it from pre-breeding migration. I developed a simple model of mortality-minimizing behaviour for migrants passing through a landscape of stopover sites to aid in understanding the tradeoffs that migrants make on post-breeding migration. The aim was not to model the entire migration, but to develop an understanding for the conditions that shape patterns of usage between sites within a region of the migratory flyway. 
% \subsection*{Migration models}

Modern migratory modelling heavily relies on two simple models: the migratory flight range equation developed by \citet{Pennycuick1975} and  ``optimal migration'' developed by \citet{Alerstam1990}. \citet{Pennycuick1975} showed that the increase in distance a migrant could fly with an additional unit of fuel declined as fuel load increased. This cost has been empirically reassessed and is lower than first thought, particularly in smaller birds such as passerines and shorebirds \citep{kvist2001Nature,Hambly2004}, but it remains an early demonstration of the fundamental tradeoff migrants must make as they load fuel for onwards migration. \citet{Alerstam1990} further identified tradeoffs on migration by categorizing migrants as attempting to minimize the time spent, energy expended, or mortality risk on migration. Each of these strategies makes specific predictions about how migrants should move across a flyway. Optimal migration theory led to the development of many migratory models that, based on migrant priorities and state, predicted optimal fuel loads, flight speeds, migratory patterns, moult strategies, and other behaviours. \citep{Weber1994,Houston1998,Weber1997,Hedenstrom1997,Farmer1998,Weber1998,Barta2006}. Many of these predictions could be tested empirically with species, groups or migrations classified as time or energy minimizing \citep{Lindstrom1992,Scheiffarth2002,Zhao2017,Duijns2009}. 

 \citet{Weber1998} were the first to develop a dynamic model of migratory movements, which many subsequently published models modify. They used the flight range equation and time-minimizing optimal strategies to develop a dynamic model that examined movements along a flyway and foraging intensities at sites as a migrant's fuel load, location, and the date changed. 

Building on optimal migration theory eventually led to modelling specific migrations. \citet{clark_fitness_1999} modelled northward migration of western sandpipers (\textit{Calidris mauri}) allowing them to optimize migratory decisions based on fuel and wind conditions. They found that the passage northward was more affected by wind patterns than by any patterns in food availability, and that a latitudinal gradient in predation danger was required to generate realistic patterns in movement.

\citet{Taylor2007} modelled individual western sandpiper movements on northward migration using a different approach. They created an individual based model in which migrants optimized their movement northward by choosing how long to stay at small and large sites, and by adjusting foraging intensity at each site. Each individual migrant optimized its behaviour based on condition as well as the behaviour of others, leading to what \citet{Taylor2007} described as a pattern of "mass-action of optimizers". Modifying the model parameters generated distinct patterns of usage at large and small sites under scenarios of increasing global predation or declining population size. They suggested the observed patterns of stopover usage could be used to distinguish between a true population decline and a change in usage due to changing conditions.

\begin{sloppypar}
\citet{bauer2006intake} developed a model of migration for pink-footed geese (\textit{Anser brachyrhynchus}) on northward migration through Norway using a state variable model based on \citet{Weber1998}. The model suggested spring phenology played a major role in migration schedule, while intake rates were of only minor importance. They used the model to suggest that hazing by farmers at stopover sites could have severe consequences for the goose population \citep{JPE:JPE1109}. Similar models were later developed to explore red knot (\textit{Calidris cantus islandica}) migratory routes \citep{Bauer2010}, consequences of habitat loss for ruddy turnstone \textit{Arenaria interpres} migration \citep{Aharon-Rotman2016}, the impact of climate-related changes to stopover sites on pink-footed geese \citep{JANE:JANE1381} and the potential for migrant geese to adjust to climate-driven changes in phenology \citep{Lameris2017}.
\end{sloppypar}

\citet{Jonker2010} modelled movements of barnacle geese \textit{Branta leucopsis} from the Netherlands toward their breeding area in Russia to examine many of the same questions explored in the pink-footed geese models. When they added predation danger, it had a strong influence on patterns of movement along the flyway compared to changes in intake rates or the advancement of spring, and easily generated realistic migratory patterns. 

\citet{brantModel2007} developed one of the few models of avian southward migration, exploring factors that shaped black brant (\textit{Branta bernicla nigricans}) migration. They modelled movement of birds from breeding grounds on the Alaska Peninsula to non-breeding grounds in Baja California or mainland Mexico using either a transoceanic flight or a longer coastal route. They predicted fuel loading rates and tail winds were key to generating an optimal strategy and that milder winters had shifted the optimal decisions for migrating geese.

The models I describe here represent by no means an exhaustive list of migratory models. \citet{Bauer2013} outline the broad array of mechanistic models that describe migration behaviour. However, other than \citet{brantModel2007}, there is a paucity of models describing post-breeding migratory behaviour, perhaps because the fitness consequences of arrival on breeding ground appear to be more standardized than the fitness function for southward migration. Models of adaptive behaviour rely on an explicit definition of expected future fitness \citep{houston1999models}.  Annual fitness functions based on migrant arrival and condition can often be estimated from empirical data of reproductive success to fledging \citep{Lameris2017}. Fitness estimation for southward migration, on the other hand, must either be assessed in an annual cycle model or by using an proxy for fitness, such as survival or mass at a given point in time \citep{Weber1999a,Alves2013a}. In this thesis I assume that adaptive behaviour on southward migration is shaped by the goal of minimizing the expected risk of mortality to the end of the migratory period.

%Realistic annual fitness functions with respect to time and condition can in many cases be estimated from empirical data, and so provide credible annual fitness payoff functions at this stage.  In contrast, maybe, for southward studies empirical data on the consequences of arriving at different times and condition for non-breeding survivorship has less often been measured.  There probably are some goose examples, if you looked hard enough, from the Netherlands.




%\subsection*{Stopover decisions in models}
%\subsection*{Stopover decisions in data}
% \subsection*{Southward migration}

% \subsection*{Southward migration in the western sandpiper}
I used the southward post-breeding migration of the western sandpiper as a focal system to explore migratory decisions. Western sandpipers migrate south from their Alaskan breeding grounds each summer, passing through a series of stopover sites before dispersing to a vast non-breeding area that stretches from Oregon to South Carolina to Peru \citep{Franks2014}. Adults depart the breeding ground prior to juveniles growing to full size, passing through stopover sites about a month ahead of juveniles, meaning their migratory passages have almost no overlap \citep{Butler1987}. The species has been well studied, particularly on migration, where detailed behavioural work has informed much of our knowledge about the tradeoffs migrant shorebirds must make on migration. Western sandpiper northward migration has been modelled twice before, first using a dynamic state variable model that examined the effects of wind patterns \citep{clark_fitness_1999} and second using an individual based model that examined how usage of small and large sites shifted under different conditions \citep{Taylor2007}. Migrant behaviour on southward migration differs, with usage spread over more sites and the migratory period being almost twice as long as that of northward migration \citep{lank_effects_2003,ydenberg_interannual_2005,Hope2011,Taylor2007}. 

The timing and behaviour of small sandpipers on southward migration appears to be strongly shaped by the migration of their primary predator, the peregrine falcon (\textit{Falco peregrinus}). Peregrine falcons migrate southward in a broad front \citep[hereafter ``falcon front''; ][]{Hope2014} with daily presence rapidly increasing in the span of a few weeks at any given location \citep{lank_effects_2003}. The timing of this migration varies substantially among years and is strongly correlated to the timing of snowmelt on breeding grounds in Alaska \citep{Niehaus2006}.  Adult sandpipers appear to truncate maternal care in order to migrate safely in front of this seasonal predator migration \citep{Jamieson2014}. Juvenile sandpipers grow to full size prior to migration \citep{Stein2006b} and often find themselves migrating alongside the falcon front as they move southward \citep{Ydenberg2007b}. 

\citet{Hope2011} describe western sandpipers as mortality-minimizers in their behaviour on southward migration. In this context mortality-minimizing behaviour means behavioural decisions that are selected for in order to minimize the probability of mortality for a migrant across the migratory period. In some cases fuelling faster may reduce overall exposure to predators, in others increasing vigilance may reduce vulnerability when predator numbers increase seasonally \citep{Hope2014}. The decisions shaped by mortality-minimization do not always refer to mortality from predators, as starvation risk can also increase mortality and often trades off against vulnerability to predation.  


%subsection*{Food/safety tradeoff}
% \subsection*{My model}
To explore stopover decisions made on southward migration by western sandpipers, I developed a model of their mortality-minimizing movement through a landscape of small and large stopover sites on a section of the flyway. The major tradeoff in the model is between small sites, which allow for higher fuel loading, but at a cost in safety, compared with larger sites, which have lower food availability, but are safer. The model was parametrized to describe the tradeoff between Sidney Island and Boundary Bay, BC: two sites about 60 kilometres apart where contrasting behavioural decisions of migratory western sandpipers have been quantified \citep[e.g. ][]{Ydenberg2002,ydenberg_western_2004}. The model is general enough that it could easily be modified to explore other flyways or areas on along this flyway, or behaviour during northward migration. 

I describe the model in detail below and present baseline results and the sensitivity  of count data to changes in model parameters. I then explore how distributions between sites shift as the relative safety and food is varied between sites. I modify the original model to explore how allowing migrants to choose their initial arrival location affects the model output. Finally, I generate patterns in usage, stopover length and masses to compare model output to available capture data from western sandpiper migration. 