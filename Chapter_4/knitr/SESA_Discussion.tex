\section{Discussion}

Over the period of 1974 - 2015 semipalmated sandpipers shifted their usage of migratory stopover sites toward larger and safer sites within the Maritimes. Across the years analysed, their stopover usage was always aggregated toward a few safe sites, but they appear to have become increasingly reliant on these very large, safe sites. Survey biases do not account for the shift across years, though the variation in distributions between years is in part explained by variation in the specific sites surveyed among years.

Semipalmated sandpipers appear to have consistently aggregated at a few sites. Aggregating in large groups has the benefits of reducing the likelihood of being selected by a predator (dilution) and increased detection of predator attacks \citep[many eyes][]{Roberts1996,Bednekoff1998,Fernandez-Juricic2007,Pays2013}. With predation dilution can also come increased competition during foraging \citep{Stillman1997,Vahl2005,Minderman2006c}. While the Bay of Fundy provides extremely rich and widespread food for refuelling sandpipers, competitive interactions can still occur at very small scales reducing foraging efficiency \citep{Vahl2005a,Beauchamp2009a,Beauchamp2014}. For most semipalmated sandpipers, the benefits of large aggregations appear to outweigh any costs to foraging efficiency.

Site safety appears to be particularly important to where the sandpipers aggregate on migration. Not only are the PMD index results suggestive of very tight aggregating or flocking at a few stopovers, but the sites they aggregate at are both large and safe. Safe sites have a larger proportion of the site further from cover, reducing the danger posed by ``stealth'' attacks that are most successful closer to cover \citep{dekker_raptor_2004}. As the number of falcons increases at stopover sites, the additional benefit that habitat safety provides, no matter how marginal, should become increasingly important. The marginal benefit to spending time in dangerous habitat has likely been reduced as predator numbers increased.

The observed shift in semipalmated sandpiper distribution matched my predicted response to increases in the continental falcon populations and the introduction of breeding pairs of peregrines around the Bay of Fundy. Because in the past the region was essentially predator free during the sandpiper migratory period, site choices could be based primarily on individual's ability to load fuel quickly at a given site. In the late 1970s, some semipalmated sandpipers were so weighted down by the large amounts of fat they had accumulated that at Kent Island, NB, they were preyed on by gulls \citep{Lank1983}. There is some evidence that fuel loads in the region have decreased at small dangerous sites such as Kent Island, but not at large sites such as Johnson's Mills \citep{hope2010influence}, likely in response to increasing predators making fat birds increasingly vulnerable to predation \citep{burns_effects_2002}. If birds are willing to forgo fuel that can be used to carry them further on their next migration flight, it is likely they are also choosing the site that optimizes the trade off between fuel loading and predation danger \citep{Pomeroy2008a,Taylor2007}. The reintroduction of breeding falcons should have had an immediate effect on site selection by sandpipers, with stopover sites around the reintroduction locations becoming immediately more dangerous. With a home range estimated to be between 123 and 1175 km$^2$ and a daily range of 23 km$^2$ around the breeding locations, the effect of falcon reintroduction should have been felt throughout the Bay of Fundy region \citep{Enderson1997a,Jenkins1998,Ganusevich2004}. 


Migrant sandpipers have previously been shown to be sensitive to predation danger on migration. Amongst the migratory behaviours that shift in response to predation include flock size, vigilance, over-ocean flocking during high tides, length of stay at dangerous sites, site selection, habitat selection within a site, and fuel load \citep{Dekker1998,ydenberg_western_2004,pomeroy_tradeoffs_2006,Pomeroy2008a,Sprague2008a}. On a longer time scale, it has been suggested semipalmated sandpipers wing length has shortened, resulting in a potential increase in escape performance as peregrine populations have recovered \citep{Lank2017}. It has also been suggested that optimally migrating sandpipers change their behaviour within seasons based on their position relative to the timing of peregrine migration \citep{Hope2014,Hope2011}. 

Changes in predation danger are not the only potential explanation for the shift in distributions. Food and danger both influence site usage in western and semipalmated sandpipers, though safety and food abundance appear to be generally negatively correlated \citep{Pomeroy2008a,Sprague2008a}. If sites were to experience an increase in food abundance uniformly across site types or if they become more uniform in their food abundance, sandpipers could find the marginal benefit of visiting a dangerous, but food abundant site to be reduced and shift towards the safer sites. \textit{Corophium volutator}, a major prey item for semipalmated sandpipers, appears to vary between sites and within season, but variation between years does not appear to be substantial \citep{Barbeau2009a}, though other studies have shown variation between years when looking at a wider array of potential food sources \citep{Quinn2012a}. It appears that the semipalmated sandpipers have a lot of flexibility in their food sources \citep{Quinn2017}, meaning that a change in one prey type abundance at a site could still be compensated by other potential food sources. I found no evidence for a longer-term trend in food overall abundance or a hypothesised change within the literature that would lead to a shift in sandpiper distributions. 

The size of the sandpiper population moving through the region could influence the distribution of birds across sites. Analyses of migratory census counts suggest there has been a sharp decrease in the population size since the 1970s, though the decline may have begun to reverse in recent years \citep{bart_survey_2007,Andres2012b,Gratto-Trevor2012}. There is also evidence from both the breeding and non-breeding grounds suggestive of a decline in the sandpiper's population \citep{Smith2012a,morrison_dramatic_2012}. Owing to biases and variation in survey effort and methodology, and due to the large area that encompasses these portions of the semipalmated sandpiper's life cycle, it has been difficult to determine with certainty the population consequences of these survey declines, but a true population decline could shape the distribution across sites under specific conditions. A reduction in the number of birds moving through a region with sites of varying suitability would decrease the density at preferred sites and any benefits from dilution across all sites and likely push birds increasingly towards the most preferred sites. If these preferred sites happened to be also the safest, the observed distribution shift should appear similar to one caused by increasing predator abundance. Simulation modelling of other migratory systems has suggested that population declines do not necessarily lead to redistributions across migratory site types \citep{Taylor2007} if migratory stopover densities of birds are below the sites' carrying capacities. The Government of Canada's index of semipalmated sandpiper population \citep{GovernmentofCanada} shows a correlated decline with my shift in the PMD index, but they are generated from the same data set, which complicates any interpretation. Repeating my analysis using data from other parts of the flyway without resident falcons might elucidate the underlying cause of the measured redistribution in bird usage. 

The number of sites surveyed, and the mean danger of the sites surveyed were the potential biases that most impacted the PMD. I estimated the impact on the index from these factors to be relatively minor (See \autoref{app:sesa}) and any impact was corrected for by the binning analysis. Binning the sites and simulating a distribution shift across sites both accounted for these biases and showed a similar shift towards safe sites. The simulations described in \autoref{app:sesa} show that the PMD can be sensitive to changes in the number of sites surveyed and the mean danger of sites surveyed, but it remains stable within the range of these parameters within the ACSS dataset we used. Nonetheless, it does highlight a potential weakness in the general use of the PMD as an index of distribution in other situations.

The PMD index was designed to account for many of the biases and lack of complete coverage within the dataset but does not include potentially important information that could help explain distributions across sites. Knowing the food abundance, lengths of stay, and refuelling rates at each site would make interpretation much simpler and allow for examining how distributions match to a true ideal free distribution as has been done in European Red Knot populations \citep{Gils2006}. The PMD index also includes the assumption that the expected distribution should be uniform in density. I consider this a reasonable baseline for semipalmated sandpipers, but for other species, such as those that exhibit strict tide-following foraging behaviour, other measures like the length of the tideline could be a better predictor of distributions. Migratory counts census bird measure usage owing to the strong influence of length of stay on the number of birds counted on a day. This fundamental caveat in conclusions drawn from migratory counts remains in our analysis and it is possible that the birds stopping at dangerous sites are just spending a shorter amount of time refuelling, as was observed in the western sandpiper \citep{ydenberg_western_2004}. Migration surveys should ideally be conducted simultaneously across a variety of site types to ensure picking up any changes in behaviour \citep{Crewe2015b}. As my exploration of the data showed, it is also important to detect any vital sites that need to be included include in the survey set each year. 

Despite some limitations, the Priority Matching Distribution index allowed us to detect a shift in the site usage across three decades of surveys. The shift is measurable and indicates a shift in distributions across sites away from dangerous sites. I believe the most likely explanation is the continental increase in peregrine populations and the reintroduction of breeding peregrines to the Bay of Fundy. Survey biases make a clear result difficult as does the correlation with the continental index for the Semipalmated population. However, the numerous examples of predator induced shifts in site choice give support for the idea of predator induced shift in distribution. The index was developed with the specific goal of understanding how birds distributed themselves across sites with respect to the sites' area and danger. The annual distributions of birds show semipalmated sandpipers are aggregated at a few sites and these sites are generally the safest of those surveyed. Recalculating the index by arranging sites in order of area of habitat instead of danger, we still see an aggregation pattern, but the largest aggregations are not at the largest sites, and there is no shift in these aggregations over time, highlighting that it is likely safety in addition to area that is influencing where sandpipers aggregate. Food abundances should also play a factor in this decision, but I lacked that information to make any specific predictions about how it should affect distributions.

In conclusion, semipalmated sandpipers aggregate in large numbers at a few large and safe sites and have increasingly shifted towards aggregating at the safest sites between 1974 and 2015. The detection of the shift highlights the potential for stopover distributions to shift, and distributions should not be assumed to be static in population trend analyses. I attribute the observed trend to a response to increases in predator populations and the reintroduction of predators into the migratory period. An increasing list of studies highlight the important role that predators play in shaping the lives of their prey, and I believe predation danger is likely influencing stopover decisions in this Maritime flyway. While biases do have the potential to affect our priority matching distribution index, it is robust to most moderate biases, and it can be used (with relevant checks of its assumption) to explore migratory distributions in a number of settings. The logical next step would be to explore other landscapes and other species to see if a similar pattern holds.