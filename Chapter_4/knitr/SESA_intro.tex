\section{Introduction}

As migratory birds move to and from their breeding grounds they encounter a matrix of potential habitats with differing characteristics for rest and refuelling. The condition of migrants and the conditions they experience will dictate the distribution of usage across potential stopover sites. Optimal migration theory predicts that as conditions change, decisions such as site choice should also change \citep{Alerstam1990}, leading to measurable redistributions across stopover sites. In this chapter I analyse counts of a small shorebird at migratory stopover sites to determine if there has been a shift in their distribution in recent decades. 

Many of the conditions that migratory birds experience on migration have changed over recent decades. Urbanization has erased or degraded habitat \citep{Iwamura2013,Studds2017}. Agricultural and sewage practices have created new habitat, but degraded others \citep{Taft2006a,Alves2012}. Climate change has begun to have impacts on many migrations, including in shorebirds, and these impacts are expected to increase in the coming century \citep{Both2007,Gordo2007,cox2010bird,Sutherland2015}. 

Birds respond to broad scale changes in a variety of ways. Urbanization has lead passerines to adjust their songs to compensate for increased urban noise \citep{Patricelli2006,Wood2006}. Vehicle traffic lead to selection on cliff swallow (\textit{Petrochelidon pyrrhonota}) wing length by differing mortality rates from vehicle collisions which decreased the average wing length in roadside adjacent populations \citep{Brown2013}. Ruffs (\textit{Calidris pugnax}) shifted their breeding range eastward, which was attributed to a decline in the quality of a migratory staging site \citep{rakhimberdiev_global_2011}. Climate change has been identified as the driver of changes in arrival on the breeding grounds and range shifts both on the breeding and non-breeding grounds \citep{Gordo2007,Jonzen2006,cox2010bird,Hu2010,Gill20132161}. Inability to keep up with phenological changes in prey availability has been identified as a major driver of population declines in some migratory bird species \citep{Both2006,Jones2010,Fraser2013a}.

Perhaps one of the most profound recent changes impacting shorebirds, amongst other groups, is the success story of North American raptor populations. From their near extinction in the 1960s, the banning of DDT and reintroduction efforts has led to a population explosion in peregrine falcon (\textit{Falco peregrinus}), Merlin (\textit{Falco columbarius}), and many other raptor populations that continues today \citep{Cade1988,Cava2014,Ydenberg2017}. Captive breeding programs helped increase the initial population recovery rate, with captive-reared falcons being released to breed at various locations throughout the continent \citep{amirault20041995,Gahbauer2015,Watts2015}. 

Prey species have responded to the continental increase in predator numbers in a variety of ways. The analogous recovery of Sea Eagle species (\textit{Haliaeetus spp.}) led prey species of seabirds at locations throughout the northern hemisphere to shift breeding locations to safer sites within a colony and to shift distributions between breeding colonies based on exposure to predators \citep{MarkHipfner2012}. The falcon population recovery drove Pacific dunlin (\textit{C. alpina pacifica}) to become more aggregated at non-breeding sites \citep{Xu2015b,Ydenberg2017}. Aggregation allows for dilution of predation danger, but can also increases competition for food resources. In years of high dunlin population counts the aggregation increased, suggesting they do not disburse as the number of birds increase, but instead aggregate at larger sites. In an example of a morphometric response to predator populations, wing lengths of semipalmated sandpiper (\textit{C. pusilla}) shortened between 1972 and 2015 as peregrine falcon populations increased, suggesting phenotypic selection to increase take-off speeds and flight agility and therefore reduce vulnerability to predation \citep{Lank2017}.

On migration, the time at stopover sites represents a large portion of migrants' total time on migration \citep{Hedenstrom1997} and can act as a period of increased vulnerability as migrants load fuel to power onwards flight \citep{Houston1998,Cimprich2005a}. As sites can vary in their predation danger based on habitat characteristics \citep{lank_ydenberg2003}, increases in predator abundances should shift usage of prey away from dangerous sites. Indeed, this is what was discovered for migrant western sandpipers (\textit{C. mauri}) at a small, food rich, but dangerous site. As their predator population recovered, the sandpipers both reduced lengths of stay and became lighter in mass \citep{ydenberg_western_2004}, showing both reduction of exposure to dangerous habitat and reduction of vulnerability to predation by reducing wing loading \citep{burns_effects_2002}. An observed trend in abundance at the site was entirely accounted for by the changes in lengths of stay. 

Census tallies at migratory stopover can be strongly affected by behavioural changes that may be hard to detect. For some shorebird species there are few ways migrants can adjust behaviour, such as those which migrate in a single leap \citep{GillJr.2009} or use only a few large staging sites \citep{Gillings2009b}. For smaller shorebird species, their behaviour appears to be more plastic, allowing them to make small adjustments to a variety of behaviours, but making a specific change harder to detect. Behavioural changes that have been detected in small sandpipers include using a variety of stopover sites, adjusting their lengths of stay, changing foraging behaviour, and choosing the stopover site based on their current condition \citep{Ydenberg2002,Beauchamp2006a,pomeroy_tradeoffs_2006,Dekker2011,Hope2014,Quinn2012a}. These results came from either detailed behavioural observations of individual birds or mark-recapture analyses. Isolating the impact a particular change in behaviour may have on survey counts is much more difficult, especially if survey methods are not well established or rigorously maintained.

One predicted behavioural change in shorebirds concerns site selection decisions. The choice between large and small sites represents a broader global tradeoff for individuals: safety versus food intake. Peregrine falcons are much deadlier when they attack from the shoreline as they can use the cover provided by shoreline habitat to mask their approach \citep{Dekker2011}, thus sites with a greater proportion of the habitat close to cover are considered more dangerous for foraging sandpipers \citep{pomeroy_experimental_2006}. There is some evidence that food is more abundant at smaller sites \citep{Pomeroy2008a,Sprague2008a}, though the generality of this pattern is uncertain. As peregrine falcon populations have recovered, the danger at all sites has increased, but the impact on the site's danger should be stronger at small, enclosed sites than at larger, more open sites \citep{Ydenberg2017}. 

The passage of the semipalmated sandpiper (\textit{C. pusilla}) through Atlantic Canada presents an opportunity to explore shifts in site usage. The semipalmated sandpiper is a small shorebird species with a hemispheric-spanning migration. Their breeding range stretches across the Arctic coast of North America and their non-breeding range across the northern coast of South America south to Peru and northern Brazil \citep{hicklin2010semipalmated}. Between these two regions migrants either pass through the interior in a series of temporary lakes and wetlands, migrate to the eastern coast of North America, or fly directly to the Caribbean coast \citep{Brown2017}. Of those that migrate to the eastern coast, the Maritimes region of Canada holds the most numerically important staging areas \citep{Hicklin87}. Migrants arrive there from the central and eastern portion of their breeding range \citep{Brown2017}, load large amounts of fuel and depart out into the Atlantic Ocean \citep{Lank1983}. A large portion of the birds using the region migrate directly to South America in a single flight of over 4000 km \citep{Lank1979,Brown2017}. Within the Maritimes region, there are numerous potential stopover sites, but the dominant location in the region is the Bay of Fundy, which has some of the world's largest tidal ranges, creating some of the largest tidal mudflats in the world \citep{garrett1972tidal}. These large flats provide safe habitat with abundant food to load fuel for onwards flight \citep{Hicklin1984,Sprague2008a,Quinn2012a}. However, many other small sites provide abundant food outside the Bay of Fundy \citep{EnvironmentCanada2009} offering migrating sandpipers a large variety of sites from which to choose.

Migrant shorebirds passing through the Maritimes are likely to be strongly influenced by both the continental peregrine population recovery and the reintroduction of breeding falcons to Atlantic Canada. Migratory peregrines do not arrive in the region until the tail end of sandpiper migration \citep{Hope2011,Worcester2008}. Between 1956 and 1981, peregrines were extirpated from the region \citep{amirault20041995}, leaving the sandpiper migratory period almost predator free. Between 1982 and 1991 a broad scale attempt to reintroduce falcons was extremely successful and the number of breeding pairs has grown exponentially. With the combination of the rapid recovery of their main predator and the reintroduction of breeding predators into a previously predator free migratory period, I predicted that semipalmated sandpipers to be increasingly likely to choose sites that are inherently safer with a lower proportion of habitat close to cover at the shoreline.

I used a survey dataset of counts of sandpipers from stopover sites to determine if semipalmated sandpipers changed their site preferences between 1974 and 2015. Based on the reintroduction of breeding falcons and the continental recover of peregrine populations, I expected to see a shift in bird usage toward safer sites as migrating birds increasingly prioritized the reductions in predation danger afforded birds utilizing safer sites. To quantify how sandpipers prioritized safety, aggregation, and habitat availability, I developed an index of the annual distribution of birds and utilized both statistical methods and simulations to diagnose whether semipalmated sandpipers adjusted stopover site selection as predator abundance has increased.